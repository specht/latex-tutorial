% !TEX program = lualatex
% ============================================================
% Demonstration mathematischer Fähigkeiten in LaTeX
% ============================================================

\documentclass[11pt,a4paper]{article}

% ------------------------------------------------------------
% Sprache & Schrift
% ------------------------------------------------------------
\usepackage{polyglossia}
\setdefaultlanguage{german}

\usepackage{fontspec}
\usepackage{unicode-math}
\setmainfont{XCharter}
\setmathfont{XCharter Math}

% ------------------------------------------------------------
% Einheiten & Zahlen
% ------------------------------------------------------------
\usepackage{siunitx}
\sisetup{
  locale = DE,
  per-mode = symbol
}

% ------------------------------------------------------------
% Layout
% ------------------------------------------------------------
\usepackage[margin=25mm]{geometry}
\setlength{\parindent}{0pt}
\setlength{\parskip}{6pt}

% ============================================================
\begin{document}

\subsection*{Mathematik im Fließtext}

Ein einfacher Term wie $3x^2 - 5x + 2$ kann problemlos im Text verwendet
werden. Auch Brüche $\frac{a}{b}$, Wurzeln $\sqrt{16} = 4$ oder Potenzen
$2^5 = 32$ sind gut lesbar.

\subsection*{Abgesetzte Formeln}

Komplexere Zusammenhänge werden übersichtlich in einer eigenen Umgebung
dargestellt:

\begin{equation}
f(x) = x^2 - 4x + 3
\end{equation}

Nullstellen lassen sich durch Faktorisierung bestimmen:

\begin{equation}
f(x) = (x - 1)(x - 3)
\end{equation}

Daraus folgen die Lösungen $x_1 = 1$ und $x_2 = 3$.

\subsection*{Summen und Indizes}

Auch mathematische Notationen wie Summen lassen sich klar darstellen:

\begin{equation}
\sum_{i=1}^{n} i = \frac{n(n+1)}{2}
\end{equation}

Diese Formel beschreibt die Summe der ersten $n$ natürlichen Zahlen.

\subsection*{Integrale}

Ein Integral kann in LaTeX ebenfalls sauber gesetzt werden, inklusive
Grenzen und Differential:

\begin{equation}
\int_0^1 x^2 \,\mathrm{d}x = \frac{1}{3}
\end{equation}

\subsection*{Gleichungssysteme}

Mehrere Gleichungen können übersichtlich angeordnet werden:

\begin{equation}
\begin{aligned}
2x + y &= 7 \\
x - y &= 1
\end{aligned}
\end{equation}

\subsection*{Einheiten im Text}

Mit dem Paket \texttt{siunitx} lassen sich Zahlen und Einheiten einheitlich
und korrekt setzen. Eine Strecke von \SI{5}{\metre} wird in
\SI{2}{\second} zurückgelegt, was einer Geschwindigkeit von
\SI{2.5}{\metre\per\second} entspricht. Ebenso lassen sich Flächen wie
\SI{15}{\centi\metre\squared} oder Zeiten von \SI{45}{\minute} ohne
manuelle Formatierung darstellen.


\end{document}
