% !TEX program = lualatex
% Dieses Dokument soll mit LuaLaTeX kompiliert werden

% --------------------------------------------------
% Dokumentklasse: KOMA-Script Briefklasse
% --------------------------------------------------
\documentclass[
  paper=a4,        % Papierformat
  fontsize=11pt,   % Grundschriftgröße
  parskip=half,    % Abstand zwischen Absätzen
  DIV=15           % Satzspiegel (Layout)
]{scrlttr2}

% --------------------------------------------------
% Spracheinstellungen
% --------------------------------------------------
% polyglossia ist die moderne Sprachverwaltung
% für LuaLaTeX und XeLaTeX
\usepackage{polyglossia}
\setdefaultlanguage{german}

% --------------------------------------------------
% Schriftarten
% --------------------------------------------------
% fontspec erlaubt die Nutzung von Systemschriften
\usepackage{fontspec}
\setmainfont{XCharter}

% --------------------------------------------------
% Mikrotypografie
% --------------------------------------------------
% microtype verbessert den Grauwert des Textes
% (Abstände, Randausgleich, Umbrüche)
\usepackage{microtype}

% --------------------------------------------------
% Farben
% --------------------------------------------------
\usepackage{xcolor}
\definecolor{accent}{HTML}{1F4E79}

% --------------------------------------------------
% Absenderdaten
% --------------------------------------------------
% Diese Variablen speichern nur Daten.
% Ob sie sichtbar sind, entscheidet das Layout.
\setkomavar{fromname}{Max Mustermann}
\setkomavar{fromaddress}{%
  Musterstraße 12\\
  12345 Musterstadt}
\setkomavar{firsthead}{
  \vspace*{5mm}\\
  Max Mustermann\\
  Telefon: +49 170 1234567\\
  E-Mail: max.mustermann@example.com\\
  Musterstraße 12,\\
  12345 Musterstadt\\[1ex]
}

% Falls dein Name nicht unter den Brief geschrieben
% werden soll, kannst du die nächste Zeile auskommentieren:
% \setkomavar{signature}{}

% --------------------------------------------------
% Betreff und Datum
% --------------------------------------------------
\setkomavar{subject}{Bewerbung um ein Schülerpraktikum im Bereich Informatik}
\setkomavar{date}{\today}

% --------------------------------------------------
% Dokumentbeginn
% --------------------------------------------------
\begin{document}

% --------------------------------------------------
% Briefumgebung
% --------------------------------------------------
\begin{letter}{%
  Beispiel GmbH\\
  Personalabteilung\\
  Beispielweg 5\\
  12345 Musterstadt}

% Anrede
\opening{Sehr geehrte Damen und Herren,}

zurzeit besuche ich die 9.\ Klasse des Mustergymnasiums in Musterstadt.
Im Rahmen der schulischen Berufsorientierung möchte ich gern ein
zweiwöchiges Schülerpraktikum absolvieren und bewerbe mich hiermit bei
Ihnen.

Besonders interessieren mich die Bereiche Informatik und Technik.
In der Schule beschäftige ich mich unter anderem mit Programmierung,
Datenbanken und praktischen Experimenten.
Außerdem dokumentiere ich meine Ergebnisse regelmäßig, zum Beispiel mit
\LaTeX.

Ihr Unternehmen ist für mich besonders interessant, da Sie praxisnah mit
moderner Software arbeiten und Einblicke in reale Arbeitsabläufe
ermöglichen.
Gern würde ich während des Praktikums Neues lernen und erste Erfahrungen
im Arbeitsalltag sammeln.

Über eine Rückmeldung oder eine Einladung zu einem Gespräch würde ich
mich sehr freuen. Anbei übersende ich Ihnen meinen Lebenslauf.

% Grußformel
\closing{Mit freundlichen Grüßen}

\end{letter}

% --------------------------------------------------
% Dokumentende
% --------------------------------------------------
\end{document}
