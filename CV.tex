% !TEX program = lualatex
% ------------------------------------------------------------
% Lebenslauf – Beispielvorlage (LuaLaTeX)
% Kompilieren mit: lualatex hermione-granger-cv.tex
%
% Optional:
% - Lege eine Datei "photo.jpg" neben diese .tex-Datei,
%   um das Platzhalter-Foto zu ersetzen.
% ------------------------------------------------------------

\documentclass[11pt,a4paper]{article}

% ------------------------------------------------------------
% Schrift & Sprache
% ------------------------------------------------------------
\usepackage{fontspec}
\setmainfont{Latin Modern Roman}
\setsansfont{Latin Modern Sans}

% ------------------------------------------------------------
% Seitenlayout
% ------------------------------------------------------------
\usepackage[a4paper,margin=2.0cm]{geometry}
\pagenumbering{gobble} % keine Seitenzahlen im Lebenslauf

% ------------------------------------------------------------
% Farben & Links
% ------------------------------------------------------------
\usepackage{xcolor}
\definecolor{accent}{HTML}{5A2E8A}
% Du kannst die Akzentfarbe hier anpassen, z. B.
% - 0D60AE für ein kräftiges Blau
% - 5A2E8A für ein dunkles Lila
% - 000000 für Schwarz

\usepackage{hyperref}
\hypersetup{
  colorlinks=true,
  linkcolor=accent,
  urlcolor=accent
}

% ------------------------------------------------------------
% Abstände, Überschriften, Listen
% ------------------------------------------------------------
\usepackage[skip=3pt]{parskip} % kompakte Absatzabstände
\usepackage{titlesec}
\usepackage{enumitem}

\titleformat{\section}
  {\large\bfseries\sffamily\color{accent}}
  {}{0pt}{}[\vspace{2pt}\hrule\vspace{3pt}]

\titlespacing*{\section}{0pt}{12pt}{9pt}

\setlist[itemize]{
  leftmargin=*,
  topsep=2pt,
  itemsep=1pt,
  parsep=0pt
}

% ------------------------------------------------------------
% Tabellen & Layout-Helfer
% ------------------------------------------------------------
\usepackage{tabularx}
\usepackage{array}
\usepackage{tabto}
\usepackage{graphicx}
\usepackage{wrapfig}

% Silbentrennung global vermeiden (sauber für CVs)
\hyphenpenalty=10000
\exhyphenpenalty=10000

% ------------------------------------------------------------
% Foto-Box (echtes Foto oder Platzhalter)
% ------------------------------------------------------------
\newcommand{\photobox}{%
  \IfFileExists{photo.jpg}{%
    \includegraphics[
      width=3.5cm,
      height=4.5cm,
      keepaspectratio
    ]{photo.jpg}%
  }{%
    \fbox{%
      \begin{minipage}[c][4.5cm][c]{3.5cm}
        \centering\small Foto
      \end{minipage}
    }%
  }%
}

% ------------------------------------------------------------
% Einheitlicher Eintrag für alle CV-Zeilen
% (linke Spalte: Zeitraum / Kategorie)
% ------------------------------------------------------------
\newcommand{\cventry}[2]{%
  \begin{tabularx}{\textwidth}
    {@{}>{\raggedright\arraybackslash\small}p{3.5cm} X@{}}
    \textbf{#1} & #2
  \end{tabularx}
  \par\vspace{2pt}
}

% ============================================================
% Dokumentbeginn
% ============================================================
\begin{document}

% ------------------------------------------------------------
% Kopfbereich mit Foto (Text läuft um das Foto herum)
% ------------------------------------------------------------
\noindent
\begin{wrapfigure}[5]{r}{4cm}
  \vspace{-12pt}
  \centering
  \photobox
\end{wrapfigure}

{\Huge\bfseries\sffamily Hermine Jean Granger}\par
\vspace{-1pt}
{\large\sffamily Schülerin}\par
\vspace{4pt}

{\sffamily
Anschrift:\tabto{2cm}27 Maple Grove, Surrey GU21 4RF, United Kingdom\\
E-Mail:\tabto{2cm}\href{mailto:h.granger@owlmail.co.uk}{h.granger@owlmail.co.uk}\\
Telefon:\tabto{2cm}+44 20 5550 0111
}

\vspace{6pt}

% ------------------------------------------------------------
% Schulische Ausbildung
% ------------------------------------------------------------
\section{Schulische Ausbildung}

\cventry
  {1991--1998}
  {%
    \textbf{Hogwarts-Schule für Hexerei und Zauberei}\newline
    \emph{Schottland}
    \begin{itemize}
      \item Schwerpunktfächer: Zauberkunst, Verteidigung, Arithmantik, Alte Runen
      \item Sehr gute Leistungen und strukturierte Prüfungsvorbereitung
      \item Aktive Mitarbeit in Lerngruppen und Tutorien
    \end{itemize}
  }

\cventry
  {bis 1991}
  {%
    \textbf{Allgemeinbildende Schule}\newline
    \emph{England}
    \begin{itemize}
      \item Sehr gute Grundlagen in Sprachen, Mathematik und Naturwissenschaften
      \item Starkes Leseinteresse und eigenständiges Lernen
    \end{itemize}
  }

% ------------------------------------------------------------
% Projekterfahrung & Engagement
% ------------------------------------------------------------
\section{Projekterfahrung \& Engagement}

\cventry
  {laufend}
  {%
    \textbf{Projektorganisation und Lernkoordination}\newline
    \emph{Schulprojekte und Arbeitsgruppen}
    \begin{itemize}
      \item Planung von Lern- und Projektphasen (Zeitpläne, Aufgabenverteilung)
      \item Recherche und Aufbereitung komplexer Inhalte
      \item Moderation von Gruppenentscheidungen
    \end{itemize}
  }

\cventry
  {projektbezogen}
  {%
    \textbf{Soziales Engagement}\newline
    \emph{Schulinitiative}
    \begin{itemize}
      \item Entwicklung von Informationsmaterial
      \item Organisation kleiner Kampagnen und Veranstaltungen
    \end{itemize}
  }

% ------------------------------------------------------------
% Kompetenzen
% ------------------------------------------------------------
\section{Kompetenzen}

\cventry
  {Stärken}
  {Analytisches Denken, sehr gute Merkfähigkeit, klare Kommunikation}

\cventry
  {Arbeitsweise}
  {Strukturiert, gewissenhaft, termintreu, eigenständig}

\cventry
  {Kenntnisse}
  {%
    \vspace*{-4mm}
    \begin{itemize}
      \item Recherche und Textarbeit
      \item Präsentation und Erklärung komplexer Inhalte
      \item Organisation und Zeitplanung
    \end{itemize}
  }

\cventry
  {Sprachen}
  {Englisch (sehr gut)}

% ------------------------------------------------------------
% Interessen
% ------------------------------------------------------------
\section{Interessen}

Lesen, Lernen, Projektarbeit, gesellschaftliches Engagement

\end{document}
