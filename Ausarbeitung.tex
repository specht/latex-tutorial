% !TEX program = lualatex
% ============================================================
% LuaLaTeX-Dokument – Schulische Ausarbeitung
% ============================================================

\documentclass[11pt,a4paper]{article}

% ------------------------------------------------------------
% Sprache
% ------------------------------------------------------------
\usepackage{polyglossia}
\setdefaultlanguage{german}

% ------------------------------------------------------------
% Layout & Typografie
% ------------------------------------------------------------
\usepackage[margin=20mm]{geometry}
\usepackage{microtype}
\usepackage{setspace}
\setstretch{1.2}

\setlength{\parindent}{0pt}
\setlength{\parskip}{6pt}

% ------------------------------------------------------------
% Schrift
% ------------------------------------------------------------
\usepackage{fontspec}
\setmainfont{XCharter}

% ------------------------------------------------------------
% Mathematik (optional)
% ------------------------------------------------------------
\usepackage{unicode-math}
\setmathfont{Latin Modern Math}

% ------------------------------------------------------------
% Tabellen & Listen
% ------------------------------------------------------------
\usepackage{tabularx}
\usepackage{enumitem}

% ------------------------------------------------------------
% Literatur
% ------------------------------------------------------------
\usepackage[
  backend=biber,
  style=authoryear,
  sorting=nyt
]{biblatex}
\addbibresource{literatur.bib}

% ------------------------------------------------------------
% Links
% ------------------------------------------------------------
\usepackage[hidelinks]{hyperref}

% ------------------------------------------------------------
% Überschriften
% ------------------------------------------------------------
\usepackage{titlesec}
\titleformat{\section}{\Large\bfseries}{\thesection}{0.8em}{}
\titleformat{\subsection}{\bfseries}{\thesubsection}{0.8em}{}

% ============================================================
\begin{document}

% ============================================================
% TITELSEITE
% ============================================================
\begin{titlepage}
\centering

\vspace*{2cm}

{\Large\bfseries Schriftliche Ausarbeitung}

\vspace{24pt}

{\huge\bfseries
Digitale Assistenzsysteme im schulischen Lernen\\
\vspace*{2mm}
Chancen, Grenzen und Perspektiven
}

\vspace{24pt}

\begin{tabular}{rl}
Name:      & Max Mustermann \\
Kurs:      & Informatik \\
Lehrkraft: & Frau Beispiel \\
Schule:    & Mustergymnasium Musterstadt \\
Schuljahr: & 2025 / 2026 \\
\end{tabular}

\vfill
Musterstadt, \today
\end{titlepage}

% ============================================================
% INHALTSVERZEICHNIS
% ============================================================
\tableofcontents
\newpage

% ============================================================
% EINLEITUNG
% ============================================================
\section{Einleitung}

Digitale Technologien prägen den schulischen Alltag zunehmend. Neben
Lernplattformen, digitalen Schulbüchern und Online-Abgaben werden auch
Systeme diskutiert, die Lernprozesse gezielt begleiten und beeinflussen
sollen.

Solche digitalen Assistenzsysteme versprechen eine individuellere
Unterstützung von Lernenden, etwa durch angepasste Aufgabenformate oder
automatisierte Rückmeldungen.\footnote{Der Begriff ist nicht eindeutig
definiert und überschneidet sich unter anderem mit „adaptiven
Lernsystemen“ oder „intelligenten Tutorensystemen“.}
Dafür werden Daten wie Bearbeitungszeiten, typische Fehler oder bisherige
Lernfortschritte ausgewertet \parencite{mueller2022}.

Ziel dieser Ausarbeitung ist es, diese Systeme im schulischen Kontext
einzuordnen und sowohl ihre Potenziale als auch ihre Grenzen aufzuzeigen.

% ============================================================
% GRUNDLAGEN
% ============================================================
\section{Grundlagen digitaler Assistenzsysteme}

Bevor konkrete Einsatzmöglichkeiten betrachtet werden, ist eine
grundlegende Klärung zentraler Begriffe und technischer Prinzipien
notwendig. Dadurch lässt sich besser einschätzen, welche Erwartungen an
digitale Assistenzsysteme realistisch sind.

\subsection{Begriffliche Einordnung}

Digitale Assistenzsysteme sind softwarebasierte Werkzeuge, die Lernende
während des Lernprozesses begleiten und unterstützen. Im Gegensatz zu
klassischen Übungsprogrammen reagieren sie auf das individuelle
Nutzungsverhalten und passen Inhalte oder Rückmeldungen entsprechend an.

\textcite{schmidt2023} beschreibt solche Systeme als datenbasierte Formen
der Lernbegleitung, die individuelle Lernpfade ermöglichen können.
Entscheidend ist dabei, dass die Funktionsweise für Lehrkräfte und
Lernende nachvollziehbar bleibt, da ansonsten Akzeptanzprobleme
entstehen können.

\subsection{Technische Grundlagen}

Technisch beruhen viele Assistenzsysteme auf vergleichsweise einfachen
Ansätzen. Häufig kommen regelbasierte Modelle oder feste
Entscheidungsstrukturen zum Einsatz, die bestimmte Reaktionen auf
typische Fehler oder Bearbeitungsmuster auslösen.

Komplexere Verfahren der künstlichen Intelligenz werden im schulischen
Kontext eher zurückhaltend eingesetzt. Neben technischen Anforderungen
spielen hier auch Fragen der Erklärbarkeit eine Rolle.

Ein Überblick über derzeit genutzte technische Ansätze findet sich unter
anderem in öffentlichen Bildungsportalen \parencite{bildungsportal2024}.

% ============================================================
% EINSATZBEREICHE
% ============================================================
\section{Mögliche Einsatzbereiche im Unterricht}

Im Unterricht eröffnen digitale Assistenzsysteme unterschiedliche
Einsatzmöglichkeiten. Sie betreffen sowohl das individuelle Lernen als
auch die Organisation und Planung von Unterricht.

\subsection{Individuelle Förderung}

Ein häufig genannter Vorteil digitaler Assistenzsysteme ist die
Möglichkeit zur individuellen Förderung. Lernende können in ihrem
eigenen Tempo arbeiten und erhalten Rückmeldungen, die auf ihre
persönlichen Schwierigkeiten zugeschnitten sind.

Gerade in Klassen mit sehr unterschiedlichen Vorkenntnissen kann dies
entlastend wirken. Während einige Schülerinnen und Schüler zusätzliche
Übungen benötigen, können andere bereits weiterführende Aufgaben
bearbeiten.

Studien legen nahe, dass adaptive Aufgabenformate insbesondere in
heterogenen Lerngruppen positive Effekte haben können
\parencite[vgl.][]{mueller2022}. Gleichzeitig zeigt sich, dass schlecht
gestaltete Rückmeldungen auch verwirrend wirken können.

\subsection{Unterstützung der Lehrkräfte}

Digitale Assistenzsysteme können auch Lehrkräfte unterstützen, etwa
durch Übersichten über den Lernstand einer Klasse oder Hinweise auf
häufige Fehler.

Diese Informationen können bei der Planung von Unterricht helfen oder
Anhaltspunkte liefern, welche Inhalte noch einmal aufgegriffen werden
sollten. Sie ersetzen jedoch keine pädagogische Entscheidung.

% ============================================================
% HERAUSFORDERUNGEN
% ============================================================
\section{Herausforderungen und Grenzen}

Neben möglichen Vorteilen bringen digitale Assistenzsysteme auch neue
Probleme und Fragestellungen mit sich, die im schulischen Kontext nicht
ignoriert werden dürfen.

\subsection{Datenschutz}

Ein zentrales Problem ist der Umgang mit personenbezogenen Daten. Da
Lernverhalten analysiert und teilweise langfristig gespeichert wird,
müssen klare Regeln zur Datenspeicherung und -verarbeitung gelten.

Besonders sensibel ist dieser Aspekt im schulischen Umfeld, da
Schülerinnen und Schüler häufig minderjährig sind und sich in einem
Abhängigkeitsverhältnis zur Institution Schule befinden
\parencite{bildungsportal2024}.

Ohne transparente Informationen darüber, welche Daten erhoben werden
und zu welchem Zweck, kann schnell Misstrauen entstehen.

\subsection{Pädagogische Verantwortung}

Digitale Systeme können Lernprozesse unterstützen, sie können jedoch
keine pädagogische Beziehung ersetzen.

Motivation, soziale Interaktion und individuelle Begleitung bleiben
zentrale Aufgaben von Lehrkräften. Assistenzsysteme sollten daher als
Ergänzung verstanden werden und nicht als Ersatz für pädagogisches
Handeln.

% ============================================================
% FAZIT
% ============================================================
\section{Fazit}

Digitale Assistenzsysteme bieten im schulischen Kontext interessante
Möglichkeiten, Lernprozesse individueller zu gestalten und Lehrkräfte
bei organisatorischen Aufgaben zu unterstützen.

Gleichzeitig zeigen sich deutliche Grenzen in Bezug auf Datenschutz,
Erklärbarkeit und pädagogische Verantwortung. Ein sinnvoller Einsatz
erfordert daher klare Rahmenbedingungen, transparente Systeme und eine
kritische Auseinandersetzung mit ihren Auswirkungen.

% ============================================================
% LITERATUR
% ============================================================
\printbibliography

\end{document}
